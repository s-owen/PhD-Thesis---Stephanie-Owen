%!TEX root = ../thesis.tex
%*******************************************************************************
%****************************** Third Chapter **********************************
%*******************************************************************************
\chapter{Quantifying and classifying tissue structure}

% **************************** Define Graphics Path **************************
\ifpdf
    \graphicspath{{Chapter3/Figs/Raster/}{Chapter3/Figs/PDF/}{Chapter3/Figs/}}
\else
    \graphicspath{{Chapter3/Figs/Vector/}{Chapter3/Figs/}}
\fi
\section{Introduction}
As shown in the previous section, stromal and epithelial immune infiltration may have differing roles within the response to a developing tumour. It is clear from observing image of tumour sections that the structure of the epithelium and stromal regions varies dramatically. These morphological differences have been discussed by Lisio et al. \cite{Lisio2019Feb} amongst others. This section of the thesis aims to set out an automated classification of such tissue structure, to investigate whether this can be reliably obtained from multiple types of tissue image and whether the structure defined this way is related to the infiltration of the tumour by particular types of immune infiltrate.  The workflow of this Chapter is illustrated in Figures \ref{fig:VA_ch4} and \ref{fig:VA_ch4_2}.

\begin{figure}
    \centering
    \includegraphics{Chapter3/Figs/Chapter3_VA.PNG}
    \caption{Visual Abstract for Chapter 4}
    \label{fig:VA_ch4}
\end{figure}

\begin{figure}
    \centering
    \includegraphics{Chapter3/Figs/VAch2_2.png}
    \caption{Visual Abstract for Chapter 4}
    \label{fig:VA_ch4_2}
\end{figure}

\section{Collaborators and Roles}
Sarwah Al-Khalidi(SAK) stained the OV04 TMA sections for CK7 and H\&E.

\section{Methodology}
\subsection{Patient Cohorts}
Data from the OV04 Cohort was used for this section as it had existing sections that had H\&E and CK staining. This cohort also had sections available for further analysis using SHG. 

\subsection{IHC}
\subsubsection{Cytokeratin 7 staining}
Tissue sections from the OV04 cohort were stained with Cytokeratin7 by Sarwah Al-Khalidi and the method of staining and imaging is laid out in section \ref{sec:sarwah_staining}.

\subsection{H\&E}
Automated Haemotoxylin and Eosin staining was carried out by SAK and is described in \ref{sec:sarwah_staining}. 

\subsection{SHG}
SHG microscopy at wavelength 850nm was performed on the Leica SP5 microscope and imaged at 20x. DAKO mounting media and a coverslip of thickness 150um was used. 

\subsection{DBSCAN}
The package \textbf{dbscan} in R was used for density based clustering analysis. The mathematics of this method is discussed in the Methods. 

\section{Results}
Having found in the previous chapter that the specific localisation of several immune infiltrates was an important factor in the prognosis of patients, I wanted to investigate the structure of tissue sections and assess the impact of structure on immune infiltrate. I wanted to use both H\&E and Cytokeratin stained images to derive this structure. Cytokeratin 7 was used as an epithelial marker to have a gold standard for structure analysis and to validate tissue classification of H\&E so that structures can be compared across multiple sections from the same core.

The morphologies discussed by Lisio et al are solid architecture, glandular architecture with slit-like spaces, papillary architecture and cribriform and pseudoendometroid architecture\cite{Lisio2019Feb} and in order to assess whether these structures could be automatically derived from images, the segmentation of tumour and stroma cells must first be acquired. 

\subsection{Patient Characteristics}
TMAs had been previously constructed from the CTCR-OV04 clinical studies, which were designed to collect imaging, blood, and tissue samples for exploratory biomarker studies. All patients provided written, informed consent for participation in these studies and for the use of their donated tissue, blood specimens, and anonymized data for the laboratory studies carried out. The CTCR-OV04 studies were approved by the Suffolk Local Research Ethics Committee (reference 05/Q0102/160) and Cambridgeshire Research Ethics Committee (reference 08/H0306/61).\cite{}

The TMA previously constructed for this cohort contained 40 patients with HGSOC. 

\begin{table}[]
    \centering
    \begin{tabular}{lc}
    \hline
       N  &  40 \\
        Age &  77 (60-90)
    \hline
    \end{tabular}
    \caption{OV04 study patient characteristics}
    \label{tab:OV04_patient}
\end{table}




\subsection{Cell classifier performance and comparison between H\&E and CK}

Images of tissue sections stained with both H\&E and a CK7 marker were analysed. QuPath was used to segment nuclei based upon their optical density. Nuclei were then classified based upon features using a random forest based classifier. Cells were split randomly into a test and training data set. Examples of tissue classification on H\&E and CK stained images. Over 5000 cells from each TMA section were labelled and 70\% were used for training and 30\% were used for validation test sets. Confusion matrices for the H\&E and CK classifier are shown in Tables (\ref{tab:classifier_he} and \ref{tab:classifier_ck}). These show a very good performance by both classifiers. 


\begin{figure}
    \centering
    \includegraphics{Figs/heck/A-2_core_class.png}
    \caption[Example of Stroma and Tumour classifier built in QuPath software]{Example of Stroma and Tumour classifier built in QuPath software. Tumour is highlighted in yellow and stroma in blue. Classifier is trained on both H\&E and CK7 stained images.}
    \label{fig:he_classify}
\end{figure}

\begin{table}[]
    \centering
    \begin{tabular}{llcc}
    \hline
    & & \multicolumn{2}{c}{Label}\\
    &   &   Stroma  &   Tumor\\
       \hline
Classification&Stroma   &    1263    &     2\\
&Tumor   &      2     &  346 \\
         
    \end{tabular}
    \caption[Confusion matrix for QuPath CK based classifier.]{Confusion matrix for QuPath CK based classifier. Classifier accuracy is 98.75\% (n=1613).}
    \label{tab:classifier_ck}
\end{table}


\begin{table}[]
    \centering
    \begin{tabular}{llcc}
    \hline
       &           &  \multicolumn{2}{c}{Label}\\
       &           &    Stroma	& Tumor\\ 
\hline
Classification & Stroma	&  2653	 &   40\\
             & Tumor	 &   11	 & 2165\\
 \hline
    \end{tabular}
    \caption[Confusion matrix for QuPath H\&E based classifier.]{Confusion matrix for QuPath H\&E based classifier. Percentage of correctly classified objects in test set: 98.95\% (n=4869).}
    \label{tab:classifier_he}
\end{table}

The classifiers both perform well and as further validation across the image types, figure\ref{fig:correlation_tumourarea} shows that the percentage of epithelial cells in each image are strongly correlated, as would be expected from serial sections. 

\begin{figure}
    \centering
    \includegraphics[width=0.5\textwidth]{Chapter3/Figs/correlation_TS_he_Ck.png}
    \caption{Correlation between percentage epithelium measured via H\&E and that measured by analysis of an image of a CK stained serial section.}
    \label{fig:correlation_tumourarea}
\end{figure}

\subsection{High Grade Serous tissue structure analysis}

In order to define a tissue structure of each core I selected several properties to examine and measure;

\begin{itemize}
    \item Stroma/Tumour percentage
    \item Surface Area: Volume ratio
    \item Cellular entropy
    \item Number of clusters 
\end{itemize}

\subsubsection*{Stroma/Tumour percentage}
This most basic measure of tissue composition was utilized in the previous chapter. In this chapter I measured this percentage as both an area and a number proportion for each image type. The distribution of epithelial cell percentages are shown in Figure \ref{fig:epi_percent}. As shown these overlap. 

\begin{figure}
    \centering
    \includegraphics[width=0.5\textwidth]{Chapter3/Figs/boxplot_epithelium.png}
    \caption{Boxplots of the distribution of epithelial percentages in the OV04 cohort based upon H\&E and CK7 stained images}
    \label{fig:epi_percentl}
\end{figure}

\subsubsection*{Invasive front}
In a purely diffusive model where immune infiltrate travels from the stroma into the epithelial nests, epithelial infiltrate would be proportional to the surface area of the epithelial nodules that are exposed to the surrounding stroma. In order to gain a measure of this surface area, I defined edge cells as those cells which had a nearest neighbour of a differing class. The surface area to volume ratio of a sphere is proportional to $\frac{1}{r}$ as is the ratio of a circumference to the area of a circle, as such the surface area of a nodule to its volume can be estimated as the following:

\[\frac{SA}{Vol} ~ \frac{(N_{\mathrm{edge}})} {(N_{\mathrm{epi}})}\]

Where $N_{\mathrm{edge}}$ is the number of cells on the edge of an epithelial cluster and $N_{\mathrm{epi}}$ is the number of cells making up the epithelial cluster.

\begin{figure}
    \centering
    \includegraphics[width=0.5\textwidth]{Chapter3/Figs/Thesis-08.png}
    \caption{Example of classified images (left) and the cells in the image designated as edge cells(right).}
    \label{fig:edge_cells}
\end{figure}

\subsubsection*{Entropy}
In addition to the number of stromal and tumour clusters, I wanted to add entropy at multiple scales to the tissue features. Entropy can be understood as a measure of mixing and the formula for deriving entropy is given in Methods \ref{eq:entropy}. I hypothesised that given that tissue regions with mixed cell types would have a higher entropy than those without, that with multiscale area sampling, a reliable classification into tumours with mixing and without could be achieved. Figure \ref{fig:entropy} gives the mean entropy over 20 subsections of the image for matched H\&E and CK stained sections. This shows a moderate correlation. 

\begin{figure}
    \centering
    \includegraphics{Chapter3/Figs/Thesis-06.png}
    \caption{Shannon entropy derived for a H\&E and CK image of serial sections from the same patient core.}
    \label{fig:entropy}
\end{figure}

\subsubsection*{Clustering with DBSCAN}
I wanted to understand how cells clustered within a tissue, and to understand the overall composition of the core, is it one solid tissue section or does it have gaps and separations. This can also be understood as whether the tumour and stroma in the core are in multiple parts and whether they are separate or well mixed.

I assessed multiple methods for clustering tissues and in terms of identifying different sizes and shapes of clusters of cells I found that it was best to use DBSCAN, a density based clustering method. An example of a results of another clustering method is shown in Figure \ref{fig:clust_bad}. K-nn clustering, for example, requires a pre-requisite number of clusters, a parameter that varies dramatically between tissue sections. K-nn is also an unsuitable method as when tissue sections comprise varying numbers within groups and varying spatial densities of cells, the method tends to split the cells equally into the pre-specified number of clusters, rather than recognising different sized clusters within.

\begin{figure}
    \centering
    \includegraphics{Chapter3/Figs/knn_example_2.png}
    \caption[k-nearest neighbour clustering of cells]{k-nearest neighbour clustering of cells identified in a tissue section. The requirement of a pre-specified number of clusters and that the algorithm does not account for varying density or information other than nearest neighbours, means that this method of clustering fails to automatically identify good clustering and in the case of densely packed TMA sections will merely split it equally.}
    \label{fig:clust_bad}
\end{figure}


The DBSCAN method requires two parameters, \textit{eps}, the distance within which two points are considered neighbours and \textit{minPts}, the minimum number of points in a cluster.

I decided to only call clusters of at least 15 cells, in tissues making up over 1500 cells, I thought each cluster should contain at least 1\% of the cells. To derive the optimum \textit{eps} for clustering groups of cells on the tissue structure level, I plotted the $k^{th}$ neighbour distribution as discussed in et.al. for $k=10$. The \textit{eps} value obtained from these was 80-85 for Epithelium and Stroma. The larger value was used for clustering all cells. These plots for tumour and stroma and the eps cutoff from each are shown in Figure \ref{fig:eps}.

\begin{figure}
    \centering
    \includegraphics[width=\textwidth]{Chapter3/Figs/Thesis-11.png}
    \caption{10th nearest neighbour plotted against index, the optimal value for the eps is at the elbow of the curve. The eps was 80 for All cells, 82 for Tumour and 85 for Stroma.}
    \label{fig:eps}
\end{figure}

An example of DBSCAN clustering on serial H\&E and CK sections is shown in Figure \ref{fig:dbclust}. These show visually accurate identifications of clusters across both samples. Distributions of the number of stroma, tumour and all cell clusters are shown in figure \ref{fig:dbscatter}.

\begin{figure}
    \centering
    \includegraphics[width=\textwidth]{Chapter3/Figs/dbscan_fullexample.png}
    \caption{Examples of clusters of cells generated with DBSCAN algorithm. Clusters generated from images of tissue sections when epithelial, stromal cells are viewed separately and when all cells are input.}
    \label{fig:dbclust}
\end{figure}

\begin{figure}
    \centering
    \includegraphics{}
    \caption{Number of clusters.}
    \label{fig:dbscatter}
\end{figure}



\subsection{Morphology classification}
Having reliably obtained all the properties above, the quantity of stromal and epithelial cells, identified subclusters, entropy, density of tissue and surface area to volume ratio of the epithelium, I aimed to then investigate whether I could classify tissue sections based upon these features into some of the morphological subtypes mentioned earlier.

I categorized sections which were over 95\% stromal cells as stromal and didn't further categorize their structure. 

The next classification we can make is to:

Following this decision tree I classified each tissue section as Solid, Papillary Glandular, Desmoplastic/Cribriform or . 

\begin{figure}
    \centering
    \includegraphics{}
    \caption{Histogram of the number of cases classified as solid, glandular, epithelial  or cribriform structure.}
    \label{fig:num_classl}
\end{figure}


\subsection{Relationship between Structure and Immune infiltration}

\subsubsection{Immune quantification}

CD8 and FOXP3 were assessed in OV04 samples.
\begin{itemize}
    \item Correlation between CD8, FOXP3 and edge cells
    \item Immune cells for different structures
\end{itemize}

\subsection{Structure and Survival}

\subsection{Speed/Quantity model}


\subsection{Collagen structure analysis}
Desmoplasia as mentioned in the previous morphology classification is the formation of fibrous stroma, one dense fibrous stroma component is Collagen. I used Second Harmonic Generation (SHG) to image Collagen on OV04 sections. SHG uses harmonic wave generation by Collagen fibres to image. Figure X shows plot of median collagen entropy and collagen energy for different classifications of tissue.

\subsection{IMC Collagen Analysis?}

\section{Discussion}

In spite of multiple data sources implying a continuous distribution of densities of infiltration of immune cells into tumours, the literature still dichotimises immune infiltrate and even categorises a subclass of tumour structures as immune reactive, an approach which fails to interpret that differences in structure may influence the immune cells which can infiltrate and the rate at which they do.

In this chapter I automated the classification of tissue morphology for the first time. 


I found that the tissue structure and surface area was correlated with immune infiltrate. 

This is an insight into how the difference in infiltration we see in patients may be due to reorganization in the tissue. 
